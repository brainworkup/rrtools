% Options for packages loaded elsewhere
\PassOptionsToPackage{unicode}{hyperref}
\PassOptionsToPackage{hyphens}{url}
\PassOptionsToPackage{dvipsnames,svgnames,x11names}{xcolor}
%
\documentclass[
  super,
  preprint,
  3p]{elsarticle}

\usepackage{amsmath,amssymb}
\usepackage{lmodern}
\usepackage{iftex}
\ifPDFTeX
  \usepackage[T1]{fontenc}
  \usepackage[utf8]{inputenc}
  \usepackage{textcomp} % provide euro and other symbols
\else % if luatex or xetex
  \usepackage{unicode-math}
  \defaultfontfeatures{Scale=MatchLowercase}
  \defaultfontfeatures[\rmfamily]{Ligatures=TeX,Scale=1}
\fi
% Use upquote if available, for straight quotes in verbatim environments
\IfFileExists{upquote.sty}{\usepackage{upquote}}{}
\IfFileExists{microtype.sty}{% use microtype if available
  \usepackage[]{microtype}
  \UseMicrotypeSet[protrusion]{basicmath} % disable protrusion for tt fonts
}{}
\makeatletter
\@ifundefined{KOMAClassName}{% if non-KOMA class
  \IfFileExists{parskip.sty}{%
    \usepackage{parskip}
  }{% else
    \setlength{\parindent}{0pt}
    \setlength{\parskip}{6pt plus 2pt minus 1pt}}
}{% if KOMA class
  \KOMAoptions{parskip=half}}
\makeatother
\usepackage{xcolor}
\setlength{\emergencystretch}{3em} % prevent overfull lines
\setcounter{secnumdepth}{5}
% Make \paragraph and \subparagraph free-standing
\ifx\paragraph\undefined\else
  \let\oldparagraph\paragraph
  \renewcommand{\paragraph}[1]{\oldparagraph{#1}\mbox{}}
\fi
\ifx\subparagraph\undefined\else
  \let\oldsubparagraph\subparagraph
  \renewcommand{\subparagraph}[1]{\oldsubparagraph{#1}\mbox{}}
\fi

\usepackage{color}
\usepackage{fancyvrb}
\newcommand{\VerbBar}{|}
\newcommand{\VERB}{\Verb[commandchars=\\\{\}]}
\DefineVerbatimEnvironment{Highlighting}{Verbatim}{commandchars=\\\{\}}
% Add ',fontsize=\small' for more characters per line
\usepackage{framed}
\definecolor{shadecolor}{RGB}{241,243,245}
\newenvironment{Shaded}{\begin{snugshade}}{\end{snugshade}}
\newcommand{\AlertTok}[1]{\textcolor[rgb]{0.68,0.00,0.00}{#1}}
\newcommand{\AnnotationTok}[1]{\textcolor[rgb]{0.37,0.37,0.37}{#1}}
\newcommand{\AttributeTok}[1]{\textcolor[rgb]{0.40,0.45,0.13}{#1}}
\newcommand{\BaseNTok}[1]{\textcolor[rgb]{0.68,0.00,0.00}{#1}}
\newcommand{\BuiltInTok}[1]{\textcolor[rgb]{0.00,0.23,0.31}{#1}}
\newcommand{\CharTok}[1]{\textcolor[rgb]{0.13,0.47,0.30}{#1}}
\newcommand{\CommentTok}[1]{\textcolor[rgb]{0.37,0.37,0.37}{#1}}
\newcommand{\CommentVarTok}[1]{\textcolor[rgb]{0.37,0.37,0.37}{\textit{#1}}}
\newcommand{\ConstantTok}[1]{\textcolor[rgb]{0.56,0.35,0.01}{#1}}
\newcommand{\ControlFlowTok}[1]{\textcolor[rgb]{0.00,0.23,0.31}{#1}}
\newcommand{\DataTypeTok}[1]{\textcolor[rgb]{0.68,0.00,0.00}{#1}}
\newcommand{\DecValTok}[1]{\textcolor[rgb]{0.68,0.00,0.00}{#1}}
\newcommand{\DocumentationTok}[1]{\textcolor[rgb]{0.37,0.37,0.37}{\textit{#1}}}
\newcommand{\ErrorTok}[1]{\textcolor[rgb]{0.68,0.00,0.00}{#1}}
\newcommand{\ExtensionTok}[1]{\textcolor[rgb]{0.00,0.23,0.31}{#1}}
\newcommand{\FloatTok}[1]{\textcolor[rgb]{0.68,0.00,0.00}{#1}}
\newcommand{\FunctionTok}[1]{\textcolor[rgb]{0.28,0.35,0.67}{#1}}
\newcommand{\ImportTok}[1]{\textcolor[rgb]{0.00,0.46,0.62}{#1}}
\newcommand{\InformationTok}[1]{\textcolor[rgb]{0.37,0.37,0.37}{#1}}
\newcommand{\KeywordTok}[1]{\textcolor[rgb]{0.00,0.23,0.31}{#1}}
\newcommand{\NormalTok}[1]{\textcolor[rgb]{0.00,0.23,0.31}{#1}}
\newcommand{\OperatorTok}[1]{\textcolor[rgb]{0.37,0.37,0.37}{#1}}
\newcommand{\OtherTok}[1]{\textcolor[rgb]{0.00,0.23,0.31}{#1}}
\newcommand{\PreprocessorTok}[1]{\textcolor[rgb]{0.68,0.00,0.00}{#1}}
\newcommand{\RegionMarkerTok}[1]{\textcolor[rgb]{0.00,0.23,0.31}{#1}}
\newcommand{\SpecialCharTok}[1]{\textcolor[rgb]{0.37,0.37,0.37}{#1}}
\newcommand{\SpecialStringTok}[1]{\textcolor[rgb]{0.13,0.47,0.30}{#1}}
\newcommand{\StringTok}[1]{\textcolor[rgb]{0.13,0.47,0.30}{#1}}
\newcommand{\VariableTok}[1]{\textcolor[rgb]{0.07,0.07,0.07}{#1}}
\newcommand{\VerbatimStringTok}[1]{\textcolor[rgb]{0.13,0.47,0.30}{#1}}
\newcommand{\WarningTok}[1]{\textcolor[rgb]{0.37,0.37,0.37}{\textit{#1}}}

\providecommand{\tightlist}{%
  \setlength{\itemsep}{0pt}\setlength{\parskip}{0pt}}\usepackage{longtable,booktabs,array}
\usepackage{calc} % for calculating minipage widths
% Correct order of tables after \paragraph or \subparagraph
\usepackage{etoolbox}
\makeatletter
\patchcmd\longtable{\par}{\if@noskipsec\mbox{}\fi\par}{}{}
\makeatother
% Allow footnotes in longtable head/foot
\IfFileExists{footnotehyper.sty}{\usepackage{footnotehyper}}{\usepackage{footnote}}
\makesavenoteenv{longtable}
\usepackage{graphicx}
\makeatletter
\def\maxwidth{\ifdim\Gin@nat@width>\linewidth\linewidth\else\Gin@nat@width\fi}
\def\maxheight{\ifdim\Gin@nat@height>\textheight\textheight\else\Gin@nat@height\fi}
\makeatother
% Scale images if necessary, so that they will not overflow the page
% margins by default, and it is still possible to overwrite the defaults
% using explicit options in \includegraphics[width, height, ...]{}
\setkeys{Gin}{width=\maxwidth,height=\maxheight,keepaspectratio}
% Set default figure placement to htbp
\makeatletter
\def\fps@figure{htbp}
\makeatother
\newlength{\cslhangindent}
\setlength{\cslhangindent}{1.5em}
\newlength{\csllabelwidth}
\setlength{\csllabelwidth}{3em}
\newlength{\cslentryspacingunit} % times entry-spacing
\setlength{\cslentryspacingunit}{\parskip}
\newenvironment{CSLReferences}[2] % #1 hanging-ident, #2 entry spacing
 {% don't indent paragraphs
  \setlength{\parindent}{0pt}
  % turn on hanging indent if param 1 is 1
  \ifodd #1
  \let\oldpar\par
  \def\par{\hangindent=\cslhangindent\oldpar}
  \fi
  % set entry spacing
  \setlength{\parskip}{#2\cslentryspacingunit}
 }%
 {}
\usepackage{calc}
\newcommand{\CSLBlock}[1]{#1\hfill\break}
\newcommand{\CSLLeftMargin}[1]{\parbox[t]{\csllabelwidth}{#1}}
\newcommand{\CSLRightInline}[1]{\parbox[t]{\linewidth - \csllabelwidth}{#1}\break}
\newcommand{\CSLIndent}[1]{\hspace{\cslhangindent}#1}

\makeatletter
\makeatother
\makeatletter
\makeatother
\makeatletter
\@ifpackageloaded{caption}{}{\usepackage{caption}}
\AtBeginDocument{%
\ifdefined\contentsname
  \renewcommand*\contentsname{Table of contents}
\else
  \newcommand\contentsname{Table of contents}
\fi
\ifdefined\listfigurename
  \renewcommand*\listfigurename{List of Figures}
\else
  \newcommand\listfigurename{List of Figures}
\fi
\ifdefined\listtablename
  \renewcommand*\listtablename{List of Tables}
\else
  \newcommand\listtablename{List of Tables}
\fi
\ifdefined\figurename
  \renewcommand*\figurename{Figure}
\else
  \newcommand\figurename{Figure}
\fi
\ifdefined\tablename
  \renewcommand*\tablename{Table}
\else
  \newcommand\tablename{Table}
\fi
}
\@ifpackageloaded{float}{}{\usepackage{float}}
\floatstyle{ruled}
\@ifundefined{c@chapter}{\newfloat{codelisting}{h}{lop}}{\newfloat{codelisting}{h}{lop}[chapter]}
\floatname{codelisting}{Listing}
\newcommand*\listoflistings{\listof{codelisting}{List of Listings}}
\makeatother
\makeatletter
\@ifpackageloaded{caption}{}{\usepackage{caption}}
\@ifpackageloaded{subcaption}{}{\usepackage{subcaption}}
\makeatother
\makeatletter
\@ifpackageloaded{tcolorbox}{}{\usepackage[many]{tcolorbox}}
\makeatother
\makeatletter
\@ifundefined{shadecolor}{\definecolor{shadecolor}{rgb}{.97, .97, .97}}
\makeatother
\makeatletter
\makeatother
\journal{Journal Name}
\makeatletter
\@ifpackageloaded{tikz}{}{\usepackage{tikz}}
\makeatother
        \newcommand*\circled[1]{\tikz[baseline=(char.base)]{
          \node[shape=circle,draw,inner sep=1pt] (char) {{\scriptsize#1}};}}  
                  
\ifLuaTeX
  \usepackage{selnolig}  % disable illegal ligatures
\fi
\usepackage[]{natbib}
\bibliographystyle{elsarticle-num}
\IfFileExists{bookmark.sty}{\usepackage{bookmark}}{\usepackage{hyperref}}
\IfFileExists{xurl.sty}{\usepackage{xurl}}{} % add URL line breaks if available
\urlstyle{same} % disable monospaced font for URLs
\hypersetup{
  pdftitle={Research Article},
  pdfauthor={Joey W. Trampush; Tommy Trojan; Biggie Smalls},
  pdfkeywords={keyword1, keyword2, keyword3},
  colorlinks=true,
  linkcolor={blue},
  filecolor={Maroon},
  citecolor={Blue},
  urlcolor={Blue},
  pdfcreator={LaTeX via pandoc}}

\setlength{\parindent}{6pt}
\begin{document}

\begin{frontmatter}
\title{Research Article \\\large{A Short Subtitle} }
\author[1]{Joey W. Trampush%
\corref{cor1}%
\fnref{fn1}}
 \ead{joey.trampush@med.usc.edu} 
\author[2]{Tommy Trojan%
%
\fnref{fn2}}
 \ead{tommy@usc.edu} 
\author[3]{Biggie Smalls%
%
\fnref{fn3}}
 \ead{biggie@badboy.com} 

\affiliation[1]{organization={USC Keck School of
Medicine, Psychiatry},addressline={Street Address},city={Los
Angeles},postcode={90033},postcodesep={}}
\affiliation[2]{organization={USC, Department Name},addressline={Street
Address},city={City},postcode={Postal Code},postcodesep={}}
\affiliation[3]{organization={National Institutes of Health, Department
Name},addressline={Street Address},city={City},postcode={Postal
Code},postcodesep={}}

\cortext[cor1]{Corresponding author}
\fntext[fn1]{This is the first author footnote.}
\fntext[fn2]{Another author footnote, this is a very long footnote and
it should be a really long footnote. But this footnote is not yet
sufficiently long enough to make two lines of footnote text.}
\fntext[fn3]{Yet another author footnote.}
        





\begin{keyword}
    keyword1 \sep keyword2 \sep 
    keyword3
\end{keyword}
\end{frontmatter}
    \ifdefined\Shaded\renewenvironment{Shaded}{\begin{tcolorbox}[frame hidden, interior hidden, boxrule=0pt, breakable, sharp corners, borderline west={3pt}{0pt}{shadecolor}, enhanced]}{\end{tcolorbox}}\fi

\textbf{Keywords:} keyword1, keyword2, keyword3

\textbf{Highlights:} These are the highlights.

\hypertarget{abstract}{%
\section{Abstract}\label{abstract}}

\hypertarget{background}{%
\subsection{Background}\label{background}}

Deserunt voluptate cupidatat officia mollit magna irure nisi. Mollit
adipisicing fugiat laboris sunt ex incididunt reprehenderit. Et velit
anim est minim ad esse excepteur officia Lorem nulla cillum est. Tempor
Lorem eu aliqua mollit et dolor.

\hypertarget{methods}{%
\subsection{Methods}\label{methods}}

Eu eu nostrud cillum duis aliquip ad do eu quis anim nostrud magna
consectetur reprehenderit ad. Pariatur adipisicing commodo velit et ut
incididunt deserunt in. Enim sint voluptate officia. Nostrud occaecat
irure qui velit tempor irure veniam nostrud enim amet labore anim.

\hypertarget{results}{%
\subsection{Results}\label{results}}

Irure enim duis pariatur magna incididunt exercitation magna ad eiusmod
aliqua do sint exercitation. Sint ea ex amet sunt sint fugiat proident
aliqua duis exercitation excepteur sit dolor. Pariatur amet Lorem ex
aliqua duis anim proident. Ad qui enim exercitation et qui nostrud velit
ad deserunt dolore occaecat do.

\hypertarget{discussion}{%
\subsection{Discussion}\label{discussion}}

Sint ad ut tempor aliqua laborum eiusmod. Sint labore nulla cillum
pariatur sit eiusmod adipisicing est ex ex irure incididunt amet et in.
Tempor ut ut velit irure ipsum do. Dolore veniam eu sit elit laborum
officia reprehenderit voluptate ut dolore.

\newpage{}

\hypertarget{introduction}{%
\section{Introduction}\label{introduction}}

Here is a citation \citep{Marwick2017}. Exercitation laborum officia
commodo mollit veniam duis. Velit mollit aliqua sunt velit. Dolor nisi
anim velit. Proident proident ea minim cillum aute. Enim fugiat in nulla
laboris aute eiusmod esse aliqua. Ullamco ullamco veniam tempor duis
aliquip in tempor reprehenderit exercitation culpa minim aliqua aliqua.
Reprehenderit esse in enim pariatur in et est eiusmod. Consectetur magna
sint labore excepteur irure id Lorem quis aute aute incididunt. Ullamco
in quis Lorem. Eu commodo pariatur culpa dolore laboris anim ipsum
labore magna.

Id officia esse culpa culpa eu cupidatat pariatur nulla. Excepteur velit
fugiat id eiusmod culpa do occaecat Lorem adipisicing amet fugiat id
exercitation aliqua qui. Qui quis eiusmod velit. Mollit ut esse occaecat
ut excepteur minim. Dolor dolore nulla aute cillum culpa ipsum esse
culpa irure mollit ipsum excepteur. In culpa voluptate elit officia
magna nostrud exercitation dolor sint quis proident occaecat ipsum. Non
dolor Lorem proident non minim cillum commodo. Pariatur ullamco fugiat
elit. Magna amet sit consectetur minim elit. Sit velit ea eu enim.

Commodo est nostrud ipsum. Sunt dolore eu tempor consectetur duis
aliquip do adipisicing non adipisicing ad do laboris. Labore incididunt
nisi et exercitation nisi veniam dolore sint ad ea Lorem exercitation
quis. Pariatur dolor aliqua est ea minim cillum pariatur cupidatat
dolor. Anim aliquip labore velit culpa ea. Voluptate cupidatat non dolor
ullamco. Anim excepteur ipsum adipisicing ut ea velit amet duis id
dolore eu. Non anim ea amet nisi veniam elit reprehenderit esse eu
consectetur est reprehenderit. Lorem nisi qui id adipisicing consectetur
fugiat nostrud ipsum tempor. Do sit voluptate qui culpa et pariatur
veniam laborum cillum aliqua dolore cupidatat.

\hypertarget{methods-1}{%
\section{Methods}\label{methods-1}}

Sit cupidatat labore culpa est deserunt sunt. Ipsum cillum mollit
officia occaecat velit voluptate ad esse reprehenderit eiusmod
adipisicing occaecat. Commodo laboris qui in culpa qui do aliquip
nostrud magna dolore. Ex anim est minim velit amet aute ullamco magna
irure labore exercitation. Consectetur pariatur cillum excepteur ullamco
officia enim nulla fugiat dolor ea ea dolor do ex.

\hypertarget{participants}{%
\subsection{Participants}\label{participants}}

Consectetur et in quis deserunt ipsum ea amet irure elit labore qui.
Nulla fugiat nostrud dolor culpa est anim laboris laboris. Aute Lorem
ullamco voluptate consectetur nulla officia laboris duis deserunt.
Consectetur eu ullamco amet pariatur pariatur non amet labore. Deserunt
excepteur sint aliqua ullamco et eu.

\hypertarget{measures}{%
\subsection{Measures}\label{measures}}

Fugiat proident amet sunt velit laborum velit. Enim id eu velit proident
cillum adipisicing. Do excepteur dolor nostrud ut exercitation aliquip
dolor anim dolore. Labore ea esse incididunt anim velit aliqua labore
consequat magna reprehenderit nostrud consequat. Sint ut Lorem Lorem
consectetur.

\hypertarget{analytical-strategy}{%
\subsection{Analytical Strategy}\label{analytical-strategy}}

Commodo quis pariatur eu Lorem veniam deserunt ut cupidatat aliquip
labore cillum. Elit et proident fugiat do tempor minim ad adipisicing
dolor magna esse labore eu consequat. Amet dolore cupidatat quis do anim
deserunt aute anim aliquip qui cillum. Ea sint velit do velit dolore est
sunt culpa irure anim. Enim ad ullamco ex sit.

\hypertarget{results-1}{%
\section{Results}\label{results-1}}

\begin{Shaded}
\begin{Highlighting}[]
\CommentTok{\# Note the path that we need to use to access our data files when rendering this document}
\NormalTok{my\_data }\OtherTok{\textless{}{-}} \FunctionTok{read.csv}\NormalTok{(here}\SpecialCharTok{::}\FunctionTok{here}\NormalTok{(}\StringTok{\textquotesingle{}src/data/raw\_data/my\_csv\_file.csv\textquotesingle{}}\NormalTok{))}
\end{Highlighting}
\end{Shaded}

\begin{Shaded}
\begin{Highlighting}[]
\FunctionTok{plot}\NormalTok{(}\FunctionTok{rnorm}\NormalTok{(}\DecValTok{10}\NormalTok{))}
\end{Highlighting}
\end{Shaded}

\begin{figure}[H]

{\centering \includegraphics{index_files/figure-pdf/fig-demo-plot-1.pdf}

}

\caption{\label{fig-demo-plot}A plot of random numbers}

\end{figure}

Figure Figure~\ref{fig-demo-plot} shows how we can have a caption and
cross-reference for a plot. Note that figure label and cross-references
must both be prefixed with \texttt{fig-}

Here is an example of inline code 3.14 in the middle of a sentence.

Here is an example of inline code 3.14 in the middle of a sentence.

\hypertarget{discussion-1}{%
\section{Discussion}\label{discussion-1}}

In sunt id laborum eiusmod fugiat in. Laboris nisi consequat ad quis
veniam voluptate consequat veniam do non nostrud. Et dolor incididunt
voluptate magna qui est ut culpa aliquip labore fugiat cupidatat in
pariatur. Labore non in deserunt est in laborum mollit in dolore laboris
eiusmod sunt. Id non id culpa nisi esse aute nulla elit proident Lorem
in sit reprehenderit. Cupidatat ut elit excepteur fugiat mollit magna eu
reprehenderit ex ad. Laboris et voluptate anim proident duis nulla
labore nisi proident anim. Culpa sit proident veniam. Et fugiat ex irure
consectetur. Laboris do quis est dolor ex fugiat sit commodo
reprehenderit ipsum exercitation excepteur excepteur pariatur enim.

\hypertarget{strengths}{%
\subsection{Strengths}\label{strengths}}

Ut labore consequat aliqua minim ullamco nisi. Et amet adipisicing
nostrud reprehenderit. Irure commodo elit consequat ipsum velit.
Voluptate sit exercitation pariatur veniam ullamco excepteur ad eu ipsum
tempor. Sint sunt anim quis do fugiat duis cupidatat enim ullamco
laboris do tempor sunt esse adipisicing. Duis qui consequat duis est
veniam adipisicing. Lorem magna ullamco veniam exercitation eiusmod esse
anim aliqua proident cupidatat in. Quis ut est dolor laboris. Proident
pariatur ad eu eiusmod. Proident enim laboris ullamco eiusmod.

\hypertarget{limitations}{%
\subsection{Limitations}\label{limitations}}

Dolor incididunt aliquip reprehenderit occaecat exercitation adipisicing
enim ea. Sit cupidatat exercitation magna cupidatat. Lorem non dolor ut.
Lorem amet esse reprehenderit reprehenderit. Fugiat anim cupidatat
incididunt cupidatat ad tempor minim cillum consectetur excepteur enim
et pariatur consectetur.

\hypertarget{acknowledgements}{%
\section{Acknowledgements}\label{acknowledgements}}

In adipisicing et ullamco ad. Proident cillum duis ea eiusmod do aliqua
laboris labore. Cupidatat tempor ad ipsum eiusmod in voluptate minim
pariatur nisi. Velit enim tempor ullamco labore. Non enim laboris ad
nulla nisi ut aliquip ut tempor fugiat laboris proident cupidatat.

\newpage{}

\hypertarget{refs}{}
\begin{CSLReferences}{0}{0}
\end{CSLReferences}

\newpage

\hypertarget{refs}{}
\begin{CSLReferences}{0}{0}
\end{CSLReferences}

\newpage{}

\hypertarget{colophon}{%
\subsubsection{Colophon}\label{colophon}}

This report was generated on 2023-01-18 12:33:27 using the following
computational environment and dependencies:

\begin{Shaded}
\begin{Highlighting}[]
\CommentTok{\# which R packages and versions?}
\ControlFlowTok{if}\NormalTok{ (}\StringTok{"devtools"} \SpecialCharTok{\%in\%} \FunctionTok{installed.packages}\NormalTok{()) devtools}\SpecialCharTok{::}\FunctionTok{session\_info}\NormalTok{()}
\end{Highlighting}
\end{Shaded}

\begin{verbatim}
- Session info ---------------------------------------------------------------
 setting  value
 version  R version 4.2.2 Patched (2023-01-06 r83584)
 os       macOS Big Sur ... 10.16
 system   x86_64, darwin17.0
 ui       X11
 language (EN)
 collate  en_US.UTF-8
 ctype    en_US.UTF-8
 tz       America/Los_Angeles
 date     2023-01-18
 pandoc   2.19.2 @ /usr/local/bin/ (via rmarkdown)

- Packages -------------------------------------------------------------------
 ! package     * version     date (UTC) lib source
 P cachem        1.0.6.9000  2022-11-30 [?] https://r-lib.r-universe.dev (R 4.2.2)
 P callr         3.7.3.9000  2022-12-24 [?] https://r-lib.r-universe.dev (R 4.2.2)
 P cli           3.6.0.9000  2023-01-11 [?] https://r-lib.r-universe.dev (R 4.2.2)
 P collapse    * 1.9.0       2023-01-15 [?] CRAN (R 4.2.0)
 P colorspace    2.1-0       2022-12-13 [?] https://r-forge.r-universe.dev (R 4.2.2)
 P crayon        1.5.2       2022-09-29 [?] CRAN (R 4.2.0)
 P data.table  * 1.14.6      2022-11-16 [?] CRAN (R 4.2.0)
 P devtools      2.4.5.9000  2022-10-11 [?] https://r-lib.r-universe.dev (R 4.2.1)
 P digest        0.6.31      2022-12-11 [?] CRAN (R 4.2.0)
 P dplyr       * 1.0.99.9000 2023-01-04 [?] repository (https://github.com/tidyverse/dplyr@dbda0c7)
 P ellipsis      0.3.2.9000  2022-12-11 [?] https://r-lib.r-universe.dev (R 4.2.2)
 P evaluate      0.20.1      2023-01-17 [?] https://r-lib.r-universe.dev (R 4.2.2)
 P fansi         1.0.3       2022-03-24 [?] CRAN (R 4.2.0)
 P fastmap       1.1.0.9000  2022-12-23 [?] https://r-lib.r-universe.dev (R 4.2.2)
 P fastverse   * 0.3.0       2022-11-15 [?] CRAN (R 4.2.0)
 P forcats     * 0.5.2.9000  2023-01-10 [?] repository (https://github.com/tidyverse/forcats@bd319e0)
 P fs            1.5.2.9000  2022-12-21 [?] https://r-lib.r-universe.dev (R 4.2.2)
 P generics      0.1.3.9000  2023-01-01 [?] https://r-lib.r-universe.dev (R 4.2.2)
 P ggplot2     * 3.4.0.9000  2023-01-06 [?] https://tidyverse.r-universe.dev (R 4.2.2)
 P glue          1.6.2.9000  2022-12-18 [?] https://tidyverse.r-universe.dev (R 4.2.2)
 P gtable        0.3.1.9000  2022-12-24 [?] https://r-lib.r-universe.dev (R 4.2.2)
 P hms           1.1.2.9002  2022-12-30 [?] https://tidyverse.r-universe.dev (R 4.2.2)
 P htmltools     0.5.4.9000  2023-01-03 [?] https://rstudio.r-universe.dev (R 4.2.2)
 P htmlwidgets   1.6.1       2023-01-07 [?] CRAN (R 4.2.0)
 P httpuv        1.6.8.9000  2023-01-12 [?] https://rstudio.r-universe.dev (R 4.2.2)
 P jsonlite      1.8.4       2022-12-06 [?] CRAN (R 4.2.0)
 P kit         * 0.0.12      2022-10-26 [?] CRAN (R 4.2.0)
 P knitr         1.41.9      2023-01-06 [?] https://yihui.r-universe.dev (R 4.2.2)
 P later         1.3.0.9000  2023-01-10 [?] https://r-lib.r-universe.dev (R 4.2.2)
 P lifecycle     1.0.3.9000  2023-01-05 [?] https://r-lib.r-universe.dev (R 4.2.2)
 P lubridate   * 1.9.0.9000  2022-12-12 [?] https://ropensci.r-universe.dev (R 4.2.2)
 P magrittr    * 2.0.3.9000  2022-12-25 [?] https://tidyverse.r-universe.dev (R 4.2.2)
 P memoise       2.0.1.9000  2023-01-03 [?] https://r-lib.r-universe.dev (R 4.2.2)
 P mime          0.12.1      2022-12-18 [?] https://yihui.r-universe.dev (R 4.2.2)
 P miniUI        0.1.1.1     2018-05-18 [?] CRAN (R 4.2.0)
 P munsell       0.5.0       2018-06-12 [?] CRAN (R 4.2.0)
 P pillar        1.8.1.9006  2023-01-01 [?] https://r-lib.r-universe.dev (R 4.2.2)
 P pkgbuild      1.4.0.9000  2022-11-27 [?] https://r-lib.r-universe.dev (R 4.2.2)
 P pkgconfig     2.0.3       2019-09-22 [?] CRAN (R 4.2.0)
 P pkgload       1.3.2.9000  2022-11-16 [?] https://r-lib.r-universe.dev (R 4.2.2)
 P prettyunits   1.1.1.9000  2022-11-30 [?] https://r-lib.r-universe.dev (R 4.2.2)
 P processx      3.8.0.9000  2022-12-18 [?] https://r-lib.r-universe.dev (R 4.2.2)
 P profvis       0.3.7.9000  2022-04-27 [?] https://rstudio.r-universe.dev (R 4.2.0)
 P promises      1.2.0.9000  2022-04-28 [?] https://rstudio.r-universe.dev (R 4.2.0)
 P ps            1.7.2.9000  2022-12-26 [?] https://r-lib.r-universe.dev (R 4.2.2)
 P purrr       * 1.0.1.9000  2023-01-10 [?] https://tidyverse.r-universe.dev (R 4.2.2)
 P R6            2.5.1.9000  2022-12-27 [?] https://r-lib.r-universe.dev (R 4.2.2)
 P Rcpp          1.0.9       2022-07-08 [?] CRAN (R 4.2.0)
 P readr       * 2.1.3.9000  2022-12-11 [?] https://tidyverse.r-universe.dev (R 4.2.2)
 P remotes       2.4.2       2021-11-30 [?] CRAN (R 4.2.0)
   renv          0.16.0-53   2023-01-13 [1] https://rstudio.r-universe.dev (R 4.2.2)
 P rlang         1.0.6.9000  2022-12-17 [?] https://r-lib.r-universe.dev (R 4.2.2)
 P rmarkdown     2.19.2      2022-12-22 [?] Github (rstudio/rmarkdown@8fabad0)
 P scales        1.2.1.9000  2023-01-01 [?] https://r-lib.r-universe.dev (R 4.2.2)
 P sessioninfo   1.2.2.9000  2022-05-14 [?] https://r~
 P shiny         1.7.4.9001  2023-01-06 [?] https://rstudio.r-universe.dev (R 4.2.2)
 P stringi       1.7.12      2023-01-11 [?] CRAN (R 4.2.2)
 P stringr     * 1.5.0.9000  2022-12-07 [?] https://tidyverse.r-universe.dev (R 4.2.2)
 P tibble      * 3.1.8.9004  2022-12-30 [?] https://tidyverse.r-universe.dev (R 4.2.2)
 P tidyr       * 1.2.1.9001  2023-01-10 [?] repository (https://github.com/tidyverse/tidyr@9174795)
 P tidyselect    1.2.0.9000  2023-01-09 [?] https://r-lib.r-universe.dev (R 4.2.2)
 P tidyverse   * 1.3.2.9000  2023-01-05 [?] repository (https://github.com/tidyverse/tidyverse@3be8283)
 P timechange    0.2.0       2023-01-11 [?] CRAN (R 4.2.0)
 P tzdb          0.3.0.9000  2023-01-04 [?] https://r-lib.r-universe.dev (R 4.2.2)
 P urlchecker    1.0.1.9000  2022-05-14 [?] https://r~
 P usethis       2.1.6.9000  2022-12-13 [?] Github (r-lib/usethis@a98a0e6)
 P utf8          1.2.2       2021-07-24 [?] CRAN (R 4.2.0)
 P vctrs         0.5.1.9000  2023-01-09 [?] Github (r-lib/vctrs@4c57418)
 P withr         2.5.0.9000  2022-12-26 [?] https://r-lib.r-universe.dev (R 4.2.2)
 P xfun          0.36.1      2023-01-12 [?] https://yihui.r-universe.dev (R 4.2.2)
 P xtable        1.8-6       2022-04-14 [?] https://r-forge.r-universe.dev (R 4.2.0)
 P yaml          2.3.6       2022-10-18 [?] CRAN (R 4.2.0)

 [1] /Users/joey/.renv/library/rrtools-8440cc86/R-4.2/x86_64-apple-darwin17.0
 [2] /Users/joey/dev/rrtools/renv/sandbox/R-4.2/x86_64-apple-darwin17.0/84ba8b13

 P -- Loaded and on-disk path mismatch.

------------------------------------------------------------------------------
\end{verbatim}

The current Git commit details are:

\begin{Shaded}
\begin{Highlighting}[]
\CommentTok{\# what commit is this file at? }
\ControlFlowTok{if}\NormalTok{ (}\StringTok{"git2r"} \SpecialCharTok{\%in\%} \FunctionTok{installed.packages}\NormalTok{() }\SpecialCharTok{\&}\NormalTok{ git2r}\SpecialCharTok{::}\FunctionTok{in\_repository}\NormalTok{(}\AttributeTok{path =} \StringTok{"."}\NormalTok{)) git2r}\SpecialCharTok{::}\FunctionTok{repository}\NormalTok{(here}\SpecialCharTok{::}\FunctionTok{here}\NormalTok{())  }
\end{Highlighting}
\end{Shaded}

\begin{verbatim}
Local:    main /Users/joey/dev/rrtools
Remote:   main @ origin (https://github.com/brainworkup/rrtools.git)
Head:     [f5f6e51] 2023-01-18: almost done
\end{verbatim}


  \bibliography{src/bib/references.bib,src/bib/bibliography.bib}


\end{document}
